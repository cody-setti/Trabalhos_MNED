\documentclass[12pt,a4paper,portuguese]{article}

\usepackage[brazil]{babel}
\usepackage[utf8]{inputenc}
\usepackage[T1]{fontenc}
\usepackage[margin=1.5cm]{geometry}
\usepackage{graphicx}
\usepackage{float}
\usepackage{tikz}
\usetikzlibrary{arrows.meta,decorations.markings,decorations.pathmorphing}
\usepackage{hyperref}
\hypersetup{
	hidelinks,
	linkcolor = blue
} % Changes the link color to black and hides the hideous red border that usually is created

\begin{document}
    % Título
    \vspace{5mm}
    \rule{0.95\textwidth}{1pt}
    \vspace{3mm}
    \begin{center}
        \textbf{\huge Equações Elípticas} \\
        \Large Trabalho 1 \\
        \large SME0202 Métodos Numéricos em Equações Diferenciais
    
        \vspace{8mm}
        \begin{tabular}{rcl}
            Cody Stefano Barham Setti &-- &4856322 \\
            Ian de Holando Cavalcante Bezerra &-- &13
        \end{tabular}

    \vspace{8mm}
    14 de Abril de 2025
    \end{center}

    \vspace{1mm}
    \rule{0.95\textwidth}{1pt}
    \vspace{0.5cm}

    % Corpo do texto
    \section{Seção}
    \subsection{Subseção}

    \begin{figure}[htbp]
        \centering
        \begin{tikzpicture}[
            > = Stealth,
            scale=1.8,
            boundary label/.style={font=\scriptsize},
            normal vector/.style={->, thick, olive, shorten >=1pt},
            point label/.style={font=\scriptsize}
        ]
            % Define coordinates
            \coordinate (A) at (0,-1);  % Bottom-left
            \coordinate (B) at (1,-1);  % Bottom-right
            \coordinate (C) at (1,1);   % Top-right
            \coordinate (D) at (0,1);   % Top-left
            \coordinate (O) at (0,0);   % Origin
            
            % Draw domain
            \fill[blue!10] (A) rectangle (C);
            
            % Draw axis (minimized)
            \draw[->, thin] (-0.25,0) -- (1.4,0) node[right, font=\scriptsize] {$x$};
            \draw[->, thin] (0,-1.25) -- (0,1.25) node[above, font=\scriptsize] {$y$};
            
            % Draw domain boundaries
            \draw[very thick] (A) -- (B) -- (C) -- (D) -- cycle;
            
            
            
            % Boundary conditions
            % Left vertical boundary (Dirichlet)
            \draw[red, very thick] (D) -- (A);
            \node[boundary label, anchor=east] at (0,0.3) {$u = 0$};
            
            % Right vertical boundary (Dirichlet)
            \draw[red, very thick] (C) -- (B);
            \node[boundary label, anchor=west] at (1,0.3) {$u = 0$};
            
            % Bottom boundary (Neumann)
            \draw[blue, very thick] (A) -- (B);
            \node[boundary label, anchor=north] at (0.5,-1.4) {$(\nabla u \cdot \mathbf{n})\vert_{(x,-1)} = e\sin(2\pi x)$};

            % Normal vectors with labels
            \draw[normal vector] (0.5,-1) -- (0.5,-1.2) node[below, olive, font=\scriptsize] {$\mathbf{n}$};
            \draw[normal vector] (0.5,1) -- (0.5,1.2) node[above, olive, font=\scriptsize] {$\mathbf{n}$};

            % Top boundary (Robin)
            \draw[blue, very thick] (D) -- (C);
            \node[boundary label, anchor=south] at (0.5,1.4) {$(\nabla u \cdot \mathbf{n} + u)\vert_{(x,1)} = 0$};
            
            % Mark coordinates with dots
            \foreach \coord/\pos/\label in {
                {(0,-1)}/below left/{(0,-1)},
                {(1,-1)}/below right/{(1,-1)},
                {(1,1)}/above right/{(1,1)},
                {(0,1)}/above left/{(0,1)},
                {(0,0)}/below right/{O}
            } {
                \fill \coord circle (1pt);
                \node[point label, \pos, inner sep=1pt] at \coord {\label};
            }
        \end{tikzpicture}
        \caption{Domínio $\Omega = [0,1] \times [-1,1]$ com suas condições de contorno}
        \label{fig:domain}
    \end{figure}
\end{document}