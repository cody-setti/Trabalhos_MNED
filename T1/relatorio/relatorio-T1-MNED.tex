\documentclass[12pt,a4paper,portuguese]{article}

\usepackage[brazil]{babel}
\usepackage[utf8]{inputenc}
\usepackage[T1]{fontenc}
\usepackage[margin=1.5cm]{geometry}
\usepackage{graphicx}
\usepackage{float}
\usepackage{tikz}
\usetikzlibrary{arrows.meta,decorations.markings,decorations.pathmorphing}
\usepackage{hyperref}
\hypersetup{
	hidelinks,
	linkcolor = blue
} % Changes the link color to black and hides the hideous red border that usually is created

\begin{document}
    % Título
    \vspace{5mm}
    \rule{0.95\textwidth}{1pt}
    \vspace{3mm}
    \begin{center}
        \textbf{\huge Equações Elípticas} \\
        \Large Trabalho 1 \\
        \large SME0224 Métodos Numéricos em Equações Diferenciais
    
        \vspace{8mm}
        \begin{tabular}{rcl}
            Cody Stefano Barham Setti &-- &4856322 \\
            Ian de Holando Cavalcante Bezerra &-- &13
        \end{tabular}

    \vspace{8mm}
    14 de Abril de 2025
    \end{center}

    \vspace{1mm}
    \rule{0.95\textwidth}{1pt}
    \vspace{0.5cm}

    % Corpo do texto
    \section{Seção}
    \subsection{Subseção}

\begin{tikzpicture}[
    > = Stealth,
    boundary condition/.style={decoration={markings, mark=at position 0.5 with {\node[black, font=\small] {#1};}}, postaction={decorate}},
    boundary arrow/.style={decoration={markings, mark=at position 0.5 with {\arrow{>}}}, postaction={decorate}}
]
    % Define coordinates
    \coordinate (A) at (0,-1);  % Bottom-left
    \coordinate (B) at (1,-1);  % Bottom-right
    \coordinate (C) at (1,1);   % Top-right
    \coordinate (D) at (0,1);   % Top-left
    
    % Draw domain
    \fill[blue!10] (A) rectangle (C);
    
    % Draw axes
    \draw[->, thick] (-0.5,0) -- (1.5,0) node[below right] {$x$};
    \draw[->, thick] (0,-1.5) -- (0,1.5) node[above left] {$y$};
    
    % Draw domain boundaries
    \draw[very thick] (A) -- (B) -- (C) -- (D) -- cycle;
    
    % Mark coordinates with dots
    \foreach \coord/\pos/\label in {
        {(0,-1)}/below left/{(0,-1)},
        {(1,-1)}/below right/{(1,-1)},
        {(1,1)}/above right/{(1,1)},
        {(0,1)}/above left/{(0,1)}
    } {
        \fill \coord circle (2pt);
        \node[\pos] at \coord {\footnotesize\label};
    }
    
    % Labels for boundaries with conditions
    \draw[boundary condition={$u=0$ (Dirichlet)}, red, very thick] (D) -- (A);
    \draw[boundary condition={$u=0$ (Dirichlet)}, red, very thick] (C) -- (B);
    \draw[boundary condition={$(\nabla u \cdot \mathbf{n})=e\sin(2\pi x)$ (Neumann)}, blue, very thick] (A) -- (B);
    \draw[boundary condition={$(\nabla u \cdot \mathbf{n} + u)=0$ (Robin)}, blue, very thick] (D) -- (C);
    
    % Domain label
    \node at (0.5,0) {$\Omega = [0,1] \times [-1,1]$};
    
    % Normal vectors
    \draw[->, thick, olive] (0.5,-1) -- (0.5,-1.3) node[below] {$\mathbf{n}$};
    \draw[->, thick, olive] (0.5,1) -- (0.5,1.3) node[above] {$\mathbf{n}$};
    \draw[->, thick, olive] (0,-0.5) -- (-0.3,-0.5) node[left] {$\mathbf{n}$};
    \draw[->, thick, olive] (1,-0.5) -- (1.3,-0.5) node[right] {$\mathbf{n}$};
    
    % Coordinate labels
    \node[below right] at (0,0) {$O$};
    \node[below] at (1,0) {$1$};
    \node[left] at (0,1) {$1$};
    \node[left] at (0,-1) {$-1$};
\end{tikzpicture}
\end{document}