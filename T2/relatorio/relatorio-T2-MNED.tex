\documentclass[11pt,a4paper,portuguese]{article}

\usepackage[brazil]{babel}
\usepackage[utf8]{inputenc}
\usepackage[T1]{fontenc}
\usepackage[margin=2cm]{geometry}
\usepackage{graphicx}
\usepackage{float}
\usepackage{tikz}
\usetikzlibrary{arrows.meta,decorations.markings,decorations.pathmorphing}
\usepackage{amsmath}
\usepackage{amssymb}
\usepackage{mathtools}

\usepackage{hyperref}
\hypersetup{
	hidelinks,
	linkcolor = blue
} % Changes the link color to black and hides the hideous red border that usually is created

\setcounter{section}{-1}

\begin{document}
    % Título
    \vspace{5mm}
    \rule{0.95\textwidth}{1pt}
    \vspace{3mm}
    \begin{center}
        \textbf{\huge Problemas de Valor Inicial} \\
        \Large Trabalho 2 \\
        \large SME0202 Métodos Numéricos em Equações Diferenciais
    
        \vspace{8mm}
        \begin{tabular}{rcl}
            Cody Stefano Barham Setti &-- &4856322 \\
            Ian de Holanda Cavalcanti Bezerra &-- &13835412
        \end{tabular}

    \vspace{8mm}
    11 de Junho de 2025
    \end{center}

    \vspace{1mm}
    \rule{0.95\textwidth}{1pt}
    \vspace{0.5cm}

    % Corpo do texto
    \section{PVI's, Soluções Numéricas e Caos}
    Neste trabalho, resolvemos numericamente dois problemas de valor inicial. O primeiro decorre da modelagem de um pêndulo simples, e, conforme nossa intuição, admite uma função periódica como solução. Já o segundo, que decorre da modelagem do pêndulo duplo, por mais que, à primeira vista, não aparente ser algo muito mais complicado que o pêndulo simples, admite, de forma geral, uma solução errática, nada periódica, muito mais complexa. Não só isso, como também pequenas perturbações neste segundo PVI levam a soluções completamente diferentes da original, um fenômeno matemático que leva o nome de `caos'.

    \section{Pêndulo Simples}
    O primeiro PVI considerado foi
    \begin{equation*}
        \begin{cases}
            q''(t) + \sin(q(t)) = 0, \\
            q(0) = q_0, \\
            q'(0) = 0.
        \end{cases}
    \end{equation*}
    O valor inicial $q'(0)=0$ foi escolhido a fim de que a EDO assumisse a solução analítica
    \begin{equation*}
        q(t) = 2\arcsin\left\{\sin\left(\frac{q_0}{2}\right)\operatorname{sn}\left[K\left(\sin^2\left(\frac{q_0}{2}\right)\right)-t;\sin^2\left(\frac{q_0}{2}\right)\right]\right\},
    \end{equation*}
    onde $\operatorname{sn}(\cdot;\cdot)$ é a função elíptica de Jacobi e $K(\cdot)$ a integral elíptica completa de primeira ordem. Tal solução analítica foi uma referência muito conveniente para estudar a consistência dos métodos numéricos utilizados.

    Por falar-se nos métodos numéricos, a EDO em sua forma apresentada, uma EDO de segunda ordem, não é compatível com eles. Antes de aplicá-los, a EDO deve ser convertida a um sistema equivalente de EDOs de primeira ordem: se definirmos $u_1(t) \coloneqq q(t)$ e $u_2(t) \coloneqq q'(t)$, então, teremos o sistema equivalente
    \begin{equation*}
        \begin{cases}
            \begin{bmatrix}u_1(t) \\ u_2(t)\end{bmatrix}' = \begin{bmatrix}u_2(t) \\ \sin(u_1(t))\end{bmatrix} \\ \\
            \begin{bmatrix}u_1(0) \\ u_2(0)\end{bmatrix} = \begin{bmatrix}q_0 \\ 0\end{bmatrix}
        \end{cases}
        \qquad\text{ou, de forma mais enxuta,}\qquad
        \begin{cases}
            \boldsymbol{u}'(t) = \boldsymbol{f}(\boldsymbol{u}(t)) \\
            \boldsymbol{u}(0) = \boldsymbol{\eta}
        \end{cases}
    \end{equation*}
    Note que este sistema é autônomo, isto é, $\boldsymbol{f}$ não depende diretamente de $t$. De toda forma, agora sim estamos em condições de aplicar nossos métodos numéricos.

    \subsection{Métodos Numéricos}
    Foram implementados três métodos numéricos para resolver o sistema de EDOs de primeira ordem equivalentes ao PVI do pêndulo simples:
    \begin{itemize}
        \item \textbf{Euler Explícito:} um método de passo único, explícito e de ordem de convergência 1;
        \item \textbf{RK4 Clássico:} um método tipo Runge-Kutta de 4 estágios, cuja tabela de Butcher é dada abaixo
        $$
        \begin{array}{c|cccc}
                  0     &       0     &                                         \\
            \frac{1}{2} & \frac{1}{2} &       0                                 \\
            \frac{1}{2} &       0     & \frac{1}{2} &       0                   \\
                  1     &       0     &       0     &       1     &       0     \\
            \hline
                        & \frac{1}{6} & \frac{1}{3} & \frac{1}{3} & \frac{1}{6} \\
        \end{array}
        $$
        \item \textbf{Euler Implícito:} um método muito similar ao `Euler Explícito', entretanto, implícito, e, portanto, de estabilidade superior. Vale ressaltar que o método de Newton para resolução de sistemas não lineares foi utilizado de forma auxiliar.
    \end{itemize}
    Vide o arquivo \verb|codigo-T2-MNED.ipynb| para os detalhes de nossa implementação de cada método.
    
    \subsection{Resolução Numérica do Pêndulo Simples}
    Vê-se no gráfico abaixo a evolução temporal fornecida por cada método para a posição angular $q(t)$ do pêndulo simples. Elegemos $q_0 = \frac{\pi}{4}\,$rad, um tempo final de $26\,$s e uma discretização temporal $h=0.01\,$s.
    \begin{figure}[H]
        \centering
        \includegraphics[width=1\textwidth]{imagens/tarefa-1-2.png}
        \caption{Evolução temporal de $q(t)$ conforme cada método}
        \label{fig:tarefa-1-1}
    \end{figure}
    Como esperado, a solução fornecida pelo método RK4 -- em vermelho -- de ordem de convergência esperada $4$, assemelha-se muito mais à solução de referência do que as fornecidas pelos métodos `Euler Explícito' -- em azul -- e `Euler Implícito' -- em verde -- ambos de ordem esperada $1$. A discrepância entre soluções é mais saliente nos tempos mais avançados, onde o acûmulo de erros é maior.

    \subsection{Ordem de Convergência}
    Abaixo, na figura da esquerda, temos um gráfico clássico de análise de ordem de convergência, onde o eixo $x$ representa o tamanho de passo temporal e o eixo $y$ o erro cometido, ambos em escala logarítmica. Calcular o erro cometido da aplicação destes métodos ao sistema do pêndulo simples foi fácil, já que temos em mãos a solução analítica: o erro foi tomado como sendo a distância máxima entre o valor da solução numérica e da exata em cada ponto da malha (a métrica induzida pela norma infinito para funções limitadas) durante um intervalo de $10\,$s, ao invés do original de $26\,$s.

    Por sua vez, na figura da direita, temos a ordem de convergência observada na transição entre malhas consecutivas. Dito de forma mais detalhada, dado duas malhas consecutivas, a mais grosseira de tamanho de passo $h_1$ e erro $e_1$ e a mais refinada de passo $h_2$ e erro $e_2$, a ordem de convergência observada na passagem da malha $1$ para a malha $2$ é $p_{\text{obs.}} = \log\left(\frac{e_1}{e_2}\right)/\log\left(\frac{h_1}{h_2}\right)$.
    \begin{figure}[H]
        \centering
        \includegraphics[width=0.9\textwidth]{imagens/tarefa-1-3.png}
        \caption{Erro máximo em função do passo de tempo para cada método.}
        \label{fig:tarefa-1-2}
    \end{figure}
    Como esperado, a ordem de convergência encontrada para os métodos `Euler Explícito' e `Euler Implícito' foi aproximadamente $1$, e, para o método RK4, aproximadamente $4$.

    \subsection{Retrato de Fase}
    A seguir, o retrato de fase de cada método, considerando-se um intervalo de $50\,$s.
    \begin{figure}[H]
        \centering
        \includegraphics[width=0.8\textwidth]{imagens/tarefa-1-4.png}
        \caption{Retrato de fase $(q,p)$ para cada método numérico.}
        \label{fig:tarefa-1-4}
    \end{figure}
    Os métodos numéricos implementados produziram retratos de fase coerentes com a teoria. O RK4,  ao menos na escala acima, possui retrato muito similar ao fornecido pela solução de referência. No entanto, o `Euler Explícito', notório por sua baixa precisão, mostra um aumento progressivo na energia do sistema, ao passo que o `Euler Implícito' mostra uma perda progressiva de energia, ambos fisicamente incondizentes com o pêndulo simples, um sistema sabidamente conservativo.
    
    \section{Pêndulo Duplo}
    O segundo PVI considerado, que advém da modelagem de um pêndulo duplo (com unidades normalizadas $m=L=g=1$), foi o seguinte
    \begin{align*}
        \begin{bmatrix}u_1(t) \\ u_2(t) \\ u_3(t) \\ u_4(t)\end{bmatrix}' = \begin{bmatrix}\frac{u_3-u_4\cos(u_1-u_2)}{2-\cos^2(u_1-u_2)} \\ \frac{2u_4-u_3\cos(u_1-u_2)}{2-\cos^2(u_1-u_2)} \\ -A_1+A_2-2\sin(u_1) \\ A_1-A_2-\sin(u_2)\end{bmatrix}, \qquad \boldsymbol{u} = \boldsymbol{\eta},
    \end{align*}
    onde $A_1 \coloneqq \frac{u_3u_4\sin(u_1-u_2)}{1+\sin^2(u_1-u_2)}$ e $A_2 \coloneqq \frac{\left[u_3^2 + 2u_4^2 - 2u_3u_4\cos(u_1-u_2)\right]\sin(u_1-u_2)\cos(u_1-u_2)}{\left[1+\sin^2(u_1-u_2)\right]^2}$.

    \subsection{Resolução Via RK4}
    Vide o arquivo \verb|relatorio-T2-MNED.ipynb| para a resolução do PVI fornecido pelo pêndulo duplo via o método RK4. Vale ressaltar que o tamanho de passo escolhido na discretização foi $h=0.005\,$s e o estado inicial escolhido para o sistema foi $\boldsymbol{\eta} = \left[\frac{2\pi}{3}\; \frac{\pi}{4}\; 0\; 0\right]^t$.

    Ademais, o \emph{solver} \verb|rk4()| implementado para a tarefa 1.2 foi reaproveitado. Apenas mudamos o valor inicial fornecido a ele e a função $\boldsymbol{f}(\boldsymbol{u}(t),t)$, que agora é correspondente ao sistema de pêndulo duplo, não ao de pêndulo simples.

    \subsection{Evolução Temporal do Pêndulo Duplo}
    O método RK4, dada a condição inicial descrita acima, forneceu o seguinte trajeto para o segundo pêndulo, durante o intervalo $\left[0\,\text{s}, 150\,\text{s}\right]$:
    \begin{figure}[H]
        \centering
        \includegraphics[width=0.75\textwidth]{imagens/tarefa-2-2.png}
        \caption{Trajetória $(x_2(t), y_2(t))$ do segundo pêndulo para condições iniciais $\boldsymbol{\eta}$}
        \label{fig:pendulo-duplo-equilibrio}
    \end{figure}

    \section{Experimentos com o Pêndulo Duplo}
    \subsection{Comportamento do Pêndulo Conforme Sua Proximidade ao Equilíbrio}
    Para o primeiro experimento, alteramos o estado inicial do sistema para
    \begin{equation*}
        \boldsymbol{\eta}' = \begin{bmatrix}0.1 \\ 0.2 \\ 0 \\ 0\end{bmatrix},
    \end{equation*}
    um estado bem mais próximo do equilíbrio, e recalculamos a trajetória do segundo pêndulo. Agora, como vê-se na figura abaixo, da esquerda, sua trajetória é quase periódica, em contraste à trajetória anterior, apresentada novamente, na figura da direita.
    
    Vale ressaltar que o eixo $x$ dos gráficos está em escalas muito dísparas.
    \begin{figure}[H]
        \centering
        \includegraphics[width=1\textwidth]{imagens/tarefa-3-a.png}
        \caption{Trajetória do segundo pêndulo próximo (esq.) e longe (dir.) da condição de equilíbrio}
        \label{fig:pendulo-duplo-caotico}
    \end{figure}

    \subsection{Pertubações ao Pêndulo Duplo e Comportamento Caótico}
    Para concluir o trabalho, fizemos mais o seguinte experimento: perturbamos sutilmente a condição inicial longe do equilíbrio, $\boldsymbol{\eta}$, resultando na nova condição inicial
    \begin{equation*}
        \boldsymbol{\eta}_{\text{pert.}} = \boldsymbol{\eta} + \boldsymbol{\delta} = \begin{bmatrix}\frac{2\pi}{3} \\ \frac{\pi}{4} \\ 0 \\ 0\end{bmatrix} + \begin{bmatrix}0.001 \\ 0 \\ 0 \\ 0\end{bmatrix}.
    \end{equation*}
    Então, comparamos as trajetórias do segundo pêndulo em cada configuração. Como vê-se na figura abaixo, por mais que a perturbação é minúscula, a trajetória do sistema perturbado rapidamente diverge da trajetória do sistema original.
    \begin{figure}[H]
        \centering
        \includegraphics[width=0.75\textwidth]{imagens/tarefa-3-b.png}
        \caption{Trajetória do segundo pêndulo na configuração original e na perturbada}
        \label{fig:pendulo-duplo-divergencia}
    \end{figure}
    É possível demonstrar que, mesmo para perturbações infinitesimais, este padrão, de trajetórias completamente discrepantes, permanecerá. Um sistema dinâmico com tal propriedade, leva o nome de `sistema caótico'. O pêndulo duplo é um sistema caótico.
\end{document}