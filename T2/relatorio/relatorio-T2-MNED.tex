\documentclass[12pt,a4paper,portuguese]{article}

\usepackage[brazil]{babel}
\usepackage[utf8]{inputenc}
\usepackage[T1]{fontenc}
\usepackage[margin=1.5cm]{geometry}
\usepackage{graphicx}
\usepackage{float}
\usepackage{tikz}
\usetikzlibrary{arrows.meta,decorations.markings,decorations.pathmorphing}
\usepackage{amsmath}
\usepackage{amssymb}
\usepackage{mathtools}

\usepackage{hyperref}
\hypersetup{
	hidelinks,
	linkcolor = blue
} % Changes the link color to black and hides the hideous red border that usually is created

\setcounter{section}{-1}

\begin{document}
    % Título
    \vspace{5mm}
    \rule{0.95\textwidth}{1pt}
    \vspace{3mm}
    \begin{center}
        \textbf{\huge Problemas de Valor Inicial} \\
        \Large Trabalho 2 \\
        \large SME0202 Métodos Numéricos em Equações Diferenciais
    
        \vspace{8mm}
        \begin{tabular}{rcl}
            Cody Stefano Barham Setti &-- &4856322 \\
            Ian de Holanda Cavalcanti Bezerra &-- &13835412
        \end{tabular}

    \vspace{8mm}
    10 de Junho de 2025
    \end{center}

    \vspace{1mm}
    \rule{0.95\textwidth}{1pt}
    \vspace{0.5cm}

    % Corpo do texto
    \section{PVI de Interesse}
    Nesse trabalho vamos resolver dois PVIs, um com um pendulo simples, o exemplo classico do osilador harmonico, e em seguida, o pendulo duplo, um exemplo de movimento chaotico.

    Vamos explorar como diferentes metodos de solucao desse PVI se comportam com diferentes fatores como malha de discretizacao ou pequenas perturbacoes as condicoes iniciais.

    \subsection{Pêndulo Simples}
    
    \begin{figure}[H]
    \centering
    \begin{tikzpicture}[scale=1.8]
        % pivot and bob
        \coordinate (O) at (0,1.5);
        \coordinate (B) at (0,0);

        % rod
        \draw (O) -- (B);

        % bob
        \draw[fill] (B) circle (0.1);

        % pivot label
        \node[above] at (O) {$O$};

        % trajectory arc: center at O, radius = 1.5, from –120° to –60°
        \draw[thick, dashed, blue]
            (O) ++(-120:1.5) arc (-120:-60:1.5);
    \end{tikzpicture}
    \caption{Pêndulo Simples}
    \label{fig:pendulof}
\end{figure}

    \subsubsection*{Formulação do Problema}
    O movimento do pêndulo simples é descrito pela seguinte equação diferencial de segunda ordem:
    \begin{equation}
        q''(t) + \sin(q(t)) = 0, \qquad q(0) = q_0, \quad q'(0) = 0
    \end{equation}
    onde $q(t)$ é o ângulo em relação à vertical. Para este trabalho, consideramos $q_0 = \frac{\pi}{4}$.

    \subsubsection*{Solução Analítica}
    A solução analítica é dada em termos de funções elípticas:
    \begin{equation}
        q(t) = 2\arcsin\left\{\sin\left(\frac{q_0}{2}\right)\operatorname{sn}\left[K\left(\sin^2\left(\frac{q_0}{2}\right)\right)-t;\sin^2\left(\frac{q_0}{2}\right)\right]\right\}
    \end{equation}
    onde $\operatorname{sn}(\cdot;\cdot)$ é a função elíptica de Jacobi e $K(\cdot)$ a integral elíptica completa de primeira ordem.

    \subsubsection*{Métodos Numéricos}
    Foram implementados três métodos numéricos para resolver o sistema equivalente de EDOs de primeira ordem:
    \begin{itemize}
        \item \textbf{Euler Explícito:} método simples de passo único, primeira ordem;
        \item \textbf{Runge-Kutta de 4ª Ordem (RK4):} método clássico de quarta ordem;
        \item \textbf{Euler Implícito:} método de primeira ordem, resolvido via método de Newton.
    \end{itemize}
    
    \subsubsection*{Resultados}
    \paragraph{Evolução Temporal}
    A seguir, apresentamos os gráficos da posição angular em função do tempo para cada método, comparando com a solução analítica de referência.
    
    % === FIGURA: Evolução Temporal ===
    \begin{figure}[H]
        \centering
        \includegraphics[width=1\textwidth]{Img_1.png} % Substitua pelo caminho correto
        \caption{Evolução temporal do ângulo $q(t)$ para cada método numérico e solução analítica.}
        \label{fig:pendulo-temporal}
    \end{figure}

    \paragraph{Ordem de Convergência}
    Os gráficos a seguir mostram o erro em função do passo de tempo para cada método, evidenciando a ordem de convergência esperada.
    
    % === FIGURA: Convergência ===
    \begin{figure}[H]
        \centering
        \includegraphics[width=1\textwidth]{Img_2.png} % Substitua pelo caminho correto
        \caption{Erro maximo em função do passo de tempo para cada método.}
        \label{fig:pendulo-convergencia}
    \end{figure}

    \paragraph{Retrato de Fase}
    O retrato de fase (velocidade angular $p$ vs. posição angular $q$) para cada método é apresentado a seguir.
    
    % === FIGURA: Retrato de Fase ===
    \begin{figure}[H]
        \centering
        \includegraphics[width=1\textwidth]{Img_3.png} % Substitua pelo caminho correto
        \caption{Retrato de fase $(q,p)$ para cada método numérico.}
        \label{fig:pendulo-fase}
    \end{figure}

    \subsubsection*{Discussão}
    Os métodos numéricos implementados produziram resultados coerentes com a teoria. O esquema de Runge–Kutta de quarta ordem (RK4) apresentou maior precisão e ordem de convergência, ao passo que os métodos de Euler explícito e implícito exibiram erros mais acentuados para passos de tempo equivalentes. No retrato de fase, fica evidente a divergência comportamental entre os métodos de Euler: o explícito acumula energia a cada oscilação em relação à solução analítica, enquanto o implícito dissipa energia, reduzindo progressivamente a amplitude das oscilações.

    
    
    % --------------------------------------------------------------------------
    \newpage
    \section{Pêndulo Duplo}

\subsection*{Formulação do Problema}
O pêndulo duplo é um sistema composto por dois pêndulos acoplados, apresentando comportamento dinâmico não-linear e, para certas condições, caótico. O sistema pode ser modelado por um conjunto de quatro EDOs de primeira ordem:
\begin{align}
    &\dot{q}_1 = p_1, \\
    &\dot{q}_2 = p_2, \\
    &\dot{p}_1 = f_1(q_1, q_2, p_1, p_2), \\
    &\dot{p}_2 = f_2(q_1, q_2, p_1, p_2)
\end{align}
As equações completas estão detalhadas no notebook e seguem a formulação clássica do problema do pêndulo duplo.

\subsection*{Método Numérico}
O sistema foi resolvido utilizando o método de Runge–Kutta de 4ª ordem (RK4), devido à sua robustez para sistemas não-lineares e capacidade de capturar comportamentos complexos como o caos.

\subsection*{Resultados}
\paragraph{Experimento 1: Condições Próximas do Equilíbrio}
Foram escolhidos valores iniciais pequenos para $q_1$ e $q_2$, próximos do equilíbrio. A trajetória do segundo pêndulo no plano $xy$ é apresentada abaixo:

% === FIGURA: Trajetória próxima do equilíbrio ===
\begin{figure}[H]
    \centering
    \includegraphics[width=0.8\textwidth]{Img_4.png} % Substitua pelo caminho correto
    \caption{Trajetória $(x_2(t), y_2(t))$ do segundo pêndulo para condições iniciais próximas do equilíbrio.}
    \label{fig:pendulo-duplo-equilibrio}
\end{figure}

Observa-se que, para pequenas amplitudes, o movimento é regular e previsível, semelhante ao de dois pêndulos simples acoplados.

\paragraph{Experimento 2: Condições Caóticas e Perturbação}
Neste experimento, foram escolhidas condições iniciais que levam a comportamento caótico. Uma pequena perturbação foi introduzida em uma das condições iniciais e as trajetórias resultantes foram comparadas:

% === FIGURA: Trajetórias caóticas ===
\begin{figure}[H]
    \centering
    \includegraphics[width=1\textwidth]{Img_5.png} % Substitua pelo caminho correto
    \caption{Trajetórias $(x_2(t), y_2(t))$ do segundo pêndulo para duas condições iniciais quase idênticas.}
    \label{fig:pendulo-duplo-caotico}
\end{figure}

Abaixo, apresentamos o gráfico da distância entre as trajetórias ao longo do tempo, evidenciando a sensibilidade às condições iniciais:

% === FIGURA: Divergência ===
\begin{figure}[H]
    \centering
    \includegraphics[width=1\textwidth]{Img_6.png} % Substitua pelo caminho correto
    \caption{Divergência entre trajetórias: distância entre $(x_2, y_2)$ para condições iniciais ligeiramente diferentes.}
    \label{fig:pendulo-duplo-divergencia}
\end{figure}

\subsection*{Discussão}
O pêndulo duplo apresenta comportamento caótico para certas condições iniciais, caracterizado por extrema sensibilidade: pequenas perturbações resultam em trajetórias completamente distintas após algum tempo. Este fenômeno é evidenciado pela rápida divergência das trajetórias no gráfico acima. O método RK4 foi essencial para capturar a dinâmica complexa do sistema, permitindo observar tanto o regime regular quanto o caótico.\par

% [Adicione comentários e discussões adicionais conforme necessário]

% [Inclua as figuras geradas pelo notebook e ajuste os caminhos conforme necessário]

\end{document}

